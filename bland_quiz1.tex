\documentclass{article}\usepackage[]{graphicx}\usepackage[]{xcolor}
% maxwidth is the original width if it is less than linewidth
% otherwise use linewidth (to make sure the graphics do not exceed the margin)
\makeatletter
\def\maxwidth{ %
  \ifdim\Gin@nat@width>\linewidth
    \linewidth
  \else
    \Gin@nat@width
  \fi
}
\makeatother

\definecolor{fgcolor}{rgb}{0.345, 0.345, 0.345}
\newcommand{\hlnum}[1]{\textcolor[rgb]{0.686,0.059,0.569}{#1}}%
\newcommand{\hlsng}[1]{\textcolor[rgb]{0.192,0.494,0.8}{#1}}%
\newcommand{\hlcom}[1]{\textcolor[rgb]{0.678,0.584,0.686}{\textit{#1}}}%
\newcommand{\hlopt}[1]{\textcolor[rgb]{0,0,0}{#1}}%
\newcommand{\hldef}[1]{\textcolor[rgb]{0.345,0.345,0.345}{#1}}%
\newcommand{\hlkwa}[1]{\textcolor[rgb]{0.161,0.373,0.58}{\textbf{#1}}}%
\newcommand{\hlkwb}[1]{\textcolor[rgb]{0.69,0.353,0.396}{#1}}%
\newcommand{\hlkwc}[1]{\textcolor[rgb]{0.333,0.667,0.333}{#1}}%
\newcommand{\hlkwd}[1]{\textcolor[rgb]{0.737,0.353,0.396}{\textbf{#1}}}%
\let\hlipl\hlkwb

\usepackage{framed}
\makeatletter
\newenvironment{kframe}{%
 \def\at@end@of@kframe{}%
 \ifinner\ifhmode%
  \def\at@end@of@kframe{\end{minipage}}%
  \begin{minipage}{\columnwidth}%
 \fi\fi%
 \def\FrameCommand##1{\hskip\@totalleftmargin \hskip-\fboxsep
 \colorbox{shadecolor}{##1}\hskip-\fboxsep
     % There is no \\@totalrightmargin, so:
     \hskip-\linewidth \hskip-\@totalleftmargin \hskip\columnwidth}%
 \MakeFramed {\advance\hsize-\width
   \@totalleftmargin\z@ \linewidth\hsize
   \@setminipage}}%
 {\par\unskip\endMakeFramed%
 \at@end@of@kframe}
\makeatother

\definecolor{shadecolor}{rgb}{.97, .97, .97}
\definecolor{messagecolor}{rgb}{0, 0, 0}
\definecolor{warningcolor}{rgb}{1, 0, 1}
\definecolor{errorcolor}{rgb}{1, 0, 0}
\newenvironment{knitrout}{}{} % an empty environment to be redefined in TeX

\usepackage{alltt}
\usepackage[sc]{mathpazo}
\renewcommand{\sfdefault}{lmss}
\renewcommand{\ttdefault}{lmtt}
\usepackage[T1]{fontenc}
\usepackage{geometry}
\geometry{verbose,tmargin=2.5cm,bmargin=2.5cm,lmargin=2.5cm,rmargin=2.5cm}
\setcounter{secnumdepth}{2}
\setcounter{tocdepth}{2}
\usepackage[unicode=true,pdfusetitle,
 bookmarks=true,bookmarksnumbered=true,bookmarksopen=true,bookmarksopenlevel=2,
 breaklinks=false,pdfborder={0 0 1},backref=false,colorlinks=false]
 {hyperref}
\hypersetup{
 pdfstartview={XYZ null null 1}}

\makeatletter
%%%%%%%%%%%%%%%%%%%%%%%%%%%%%% User specified LaTeX commands.
\renewcommand{\textfraction}{0.05}
\renewcommand{\topfraction}{0.8}
\renewcommand{\bottomfraction}{0.8}
\renewcommand{\floatpagefraction}{0.75}

\makeatother
\IfFileExists{upquote.sty}{\usepackage{upquote}}{}
\begin{document}








The results below are generated from an R script.

\begin{knitrout}
\definecolor{shadecolor}{rgb}{0.969, 0.969, 0.969}\color{fgcolor}\begin{kframe}
\begin{alltt}
---
  title: \hlsng{"Quiz 1: Part 2"}
format: html
editor: visual
embed-resources: true
---

\hlcom{  ## Part 2: Coding in action}

\hlcom{  ### Instructions}

  Please answer these questions in R using whatever syntax makes sense to you. Feel free to use the R help pages and/or notes from class.

**If you use Google or ChatGPT etc. to help you get to an answer, please note so on your answer**

  Submit your answers as an html file. Name it \hlsng{"lastname_quiz1.html"}. Two submission options:\textbackslash{}
1. Email to mcnew\textbackslash{}@arizona.edu 2. Create a branch in your forked-and-cloned class repo. Add the html file to the student_contributions/ folder. Submit a pull request to \hlkwd{Sabrina} (extra credit).

```\{r, message = F\}
\hlcom{# load packages and data}
\hlkwd{library}(dslabs)
\hlkwd{library}(dplyr)
```

\hlcom{#### Load your data:}

Load the *swallows.csv* into R, call it swallows. This data frame lists results of an experiment on tree swallows. Each row is a nest. Nests were put into one of two **treatments**: simulated \hlkwd{predation} (predator), or \hlkwd{control} (control). Along with treatment you have the following columns:\textbackslash{}
**nest_fate** = whether the nest fledged any young or whether they all died\textbackslash{}
**brood** = number of nestlings\textbackslash{}
**n_fledged** = number of nestlings that fledged.

You also have a separate data frame called *brightness.csv* where each row lists the plumage \hlkwd{brightness} (a sexual signal) for the female at each nest. Load this one up too.

```\{r\}
\hlcom{# Answer}
swallows <- \hlkwd{read.csv}(\hlsng{'swallows.csv'})
brightness <- \hlkwd{read.csv}(\hlsng{'brightness.csv'})

\hlcom{# Google? y/n [n]}
```

\hlcom{#### Wrangle your data}

You realize that you want to add brightness information onto your main swallows data frame. Join these data frames using nest_id as your key variable. Describe how you would check to make sure this join worked as intended.

```\{r\}
\hlcom{#Answer}

bright_swa <- \hlkwd{merge}(swallows, brightness, by=\hlsng{'nestbox'})

\hlcom{# Google? y/n [y]}
```

Inspect your combined data frame. Do some sort of quick sanity check on each column to catch any obvious entry errors. Correct as necessary

```\{r\}
\hlcom{# Answer}

bright_swa <- \hlkwd{na.omit}(bright_swa)

\hlcom{# Google y/n [n]}
```

Create a new column called prop_fledged that contains the proportion of nestlings that fledged from each nest.

```\{r\}
\hlcom{#Answer}

bright_swa <- bright_swa %>%
  \hlkwd{mutate}(prop_fledged = n_fledged/brood) %>%
  \hlkwd{mutate_all}(~\hlkwd{replace}(., \hlkwd{is.nan}(.), 0))


\hlcom{# Google? y/n [y for dropping NaNs]}
```

\hlcom{#### Explore the data}

What was the mean proportion of nestlings fledged in each treatment?

```\{r\}
\hlcom{# Answer}

bright_swa %>%
  \hlkwd{group_by}(treatment) %>%
  \hlkwd{summarize}(average = \hlkwd{mean}(prop_fledged))

\hlcom{# Google? y/n [n]}
```

Create a plot showing fledging success differences between treatments

```\{r\}
\hlcom{#Answer}

\hlkwd{library}(ggplot2)
fledg_plot <- bright_swa %>%
  \hlkwd{ggplot}(\hlkwd{aes}(treatment, prop_fledged)) +
  \hlkwd{geom_violin}()

\hlcom{# Google? y/n [n]}
```

Is there any evidence that female brightness influenced fledging success? Create a plot to visually investigate this question.

```\{r\}
\hlcom{#Answer}
\hlcom{# There seems to be no relationhsip between brightness and fledging success.The}
\hlcom{# graph depicts no discernable impact of brightness on fledging success.}

bright_plot <- bright_swa %>%
  \hlkwd{ggplot}(\hlkwd{aes}(brightness, prop_fledged, color = treatment)) +
  \hlkwd{geom_jitter}()

\hlcom{# Google? y/n [n]}
```
\end{alltt}


{\ttfamily\noindent\bfseries\color{errorcolor}{\#\# Error: <text>:12:10: unexpected symbol\\\#\# 11: \\\#\# 12: \ \ Please answer\\\#\# \ \ \ \ \ \ \ \ \ \ \ \ \ \textasciicircum{}}}\end{kframe}
\end{knitrout}

The R session information (including the OS info, R version and all
packages used):

\begin{knitrout}
\definecolor{shadecolor}{rgb}{0.969, 0.969, 0.969}\color{fgcolor}\begin{kframe}
\begin{alltt}
\hlkwd{sessionInfo}\hldef{()}
\end{alltt}
\begin{verbatim}
## R version 4.4.1 (2024-06-14 ucrt)
## Platform: x86_64-w64-mingw32/x64
## Running under: Windows 11 x64 (build 22621)
## 
## Matrix products: default
## 
## 
## locale:
## [1] LC_COLLATE=English_United States.utf8  LC_CTYPE=English_United States.utf8   
## [3] LC_MONETARY=English_United States.utf8 LC_NUMERIC=C                          
## [5] LC_TIME=English_United States.utf8    
## 
## time zone: America/Phoenix
## tzcode source: internal
## 
## attached base packages:
## [1] stats     graphics  grDevices utils     datasets  methods   base     
## 
## other attached packages:
## [1] knitr_1.47    ggplot2_3.5.1 dplyr_1.1.4   dslabs_0.8.0 
## 
## loaded via a namespace (and not attached):
##  [1] crayon_1.5.3      vctrs_0.6.5       nlme_3.1-164      cli_3.6.3        
##  [5] xfun_0.45         rlang_1.1.4       highr_0.11        generics_0.1.3   
##  [9] glue_1.7.0        labeling_0.4.3    colorspace_2.1-0  scales_1.3.0     
## [13] fansi_1.0.6       grid_4.4.1        evaluate_0.24.0   munsell_0.5.1    
## [17] tibble_3.2.1      lifecycle_1.0.4   compiler_4.4.1    pkgconfig_2.0.3  
## [21] mgcv_1.9-1        rstudioapi_0.16.0 lattice_0.22-6    farver_2.1.2     
## [25] R6_2.5.1          tidyselect_1.2.1  utf8_1.2.4        pillar_1.9.0     
## [29] splines_4.4.1     magrittr_2.0.3    Matrix_1.7-0      tools_4.4.1      
## [33] withr_3.0.0       gtable_0.3.5
\end{verbatim}
\begin{alltt}
\hlkwd{Sys.time}\hldef{()}
\end{alltt}
\begin{verbatim}
## [1] "2024-10-08 03:44:05 MST"
\end{verbatim}
\end{kframe}
\end{knitrout}


\end{document}
